Eine Zustandsmaschine ist für ein kleines System eine passende Vorgehensweise. Für komplexere Systeme sollte eine andere Art von der Beschreibung des Systems gewählt werden, da für jeden Zustand ein neuer Wert definiert werden muss und dieser in das 'Switch-Case Statement' eingefügt werden muss. Dies führt bei komplexeren Systemen zu einer Verschlechterung der Code-Qualität und einem unleserlicheren Quellcode. Der Aufgabenstellung konnte entsprochen werden. So wurde ein Nummernschloss entworfen, welches entsprechend der Zustandsmaschine arbeitet und diese Zustände angemessen visualisiert. Auch wurde das Makefile so verändert, dass diese für den benutzen Mikrokontroller (Arduino Uno - ATMega328p) benutzt werden konnte.

Die Verwendung der Open-Source-Software 'Arduino-IDE' \footnote{\url{https://github.com/arduino/Arduino}} wäre aufgrund von weitreichender Unterstützung von Arduino-Mikrocontrollern eine bessere Entscheidung gewesen als die 'Eclipse'-IDE. So gab es am Anfang Probleme bei der Einrichtung der Entwicklungsumgebung, sowie Unstimmigkeiten bei einigen Funktionen. Dennoch konnte die IDE 'Eclipse' für dieses Projekt benutzt werden um den Quellcode zu erarbeiten und ein Makefile zu erstellen, sowie zum Hochladen der Anwendung auf den Mikrocontroller.

