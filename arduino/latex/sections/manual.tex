Sobald das Gerät mit Strom versorgt wird, kann die Schlosssteuerung benutzt werden. Hierfür visualisiert das grüne Leuchten der LED, dass die Schlosssteuerung korrekt gestartet wurde und auf einen Tastendruck wartet. Die Initial-PIN für das Entsperren des Schlosses lautet 'a123'. Dieser kann nach korrekter Eingabe geändert werden. Wenn eine Taste gedrückt wird, leuchtet die LED kurz Gelb auf um den Tastendruck zu bestätigen. Bei Misserfolg wird das Schloss zurückgesetzt und die vollständige PIN muss nochmals eingegeben werden. Wurde die PIN korrekt eingegeben wird das Schloss freigegeben und die LED leuchtet dauerhaft grün. Hier gibt es nun die Möglichkeit, die PIN zu ändern. Mit der Tastenkombination '\#a' im freigegebenen Schloss kann das Ändern der PIN initiiert werden. Nun leuchtet die LED Türkis und eine neue 4-stellige PIN kann eingegeben werden. Nach der Eingabe der neuen PIN wechselt das Schloss wieder in den geöffneten Modus. Nach einem kurzen Zeitintervall ohne Benutzereingabe wird das Schloss wieder geschlossen und die PIN muss erneut eingegeben werden. Sollte der PIN dreimal falsch eingegeben worden sein, wird das Schloss gesperrt und muss mit einem PUK entsperrt werden. Der PUK für das Entsperren lautet '\#12345'. Dieser kann beliebig oft eingegeben werden und erst bei erfolgreicher Eingabe wird das Schloss entsperrt.