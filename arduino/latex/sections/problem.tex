Das Ziel des Projektes ist die Realisierung eines Nummernschlosses mit LED Visualisierung, welches die folgenden Funktionen erfüllen muss: 
\begin{itemize}
    \item Schlüsseleingabe mit PIN bestehend aus den Zeichen 0 bis 9 und A bis D
    \item Visualisierung erfolgreicher Schlüsseleingabe durch eine LED
    \item Nach erfolgreicher Schlüsseleingabe kann ein neuer PIN festgelegt werden
    \item Nach einer festgelegten Anzahl von Misserfolgen während der PIN-Eingabe soll das Schloss gesperrt werden
    \item Wenn das Schloss gesperrt ist, kann es mit einem PUK wieder entsperrt werden
\end{itemize}

Dieses Projekt soll mit einem 'Arduino UNO'-Mikrocontroller realisiert werden, ohne zusätzliche Einbindung anderer Bibliotheken mit Ausnahme der Standardbibliothek. Die Implementierung soll mit der Programmiersprache C unter Verwendung der IDE Eclipse vollzogen werden. Das Projekt soll zudem modular aufgebaut werden, sowie vollständig dokumentiert sein. Dies beinhaltet die vollständige Dokumentation der für den Betrieb der Sensoren/Aktoren relevanten Funktionen im Quellcode (Doxygen-Dokumentation). 